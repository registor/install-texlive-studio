% 加载宏包
%===================注意======================%
% 在调用beamer.cls宏包后,以下宏包将自动调用,
% 不应单独调用这些宏包,以免发生冲突
% amsfonts, amsmath, amssymb, amsthm, 
% enumerate, geometry, graphics, graphicx, 
% hyperref, url, 
% ifpdf, keyval, xcolor, xxcolor
% =============================================%
% % 调整行间距
\usepackage{setspace}

\usepackage{fontawesome5}

\usepackage{unicode-math}

\usepackage{qrcode}

\usepackage{multicol}

\usepackage{csquotes}

% ========排版键盘组合和菜单的宏包=========
\usepackage{menukeys}

% 代码排版工具宏包
% 需要预先安装 python 和 pygments。
% 此宏包要加  -shell-escape 编译参数
% 版本2.0,支持行内代码排版
\usepackage{minted}

% 特殊符号
\usepackage{pifont}

% 代码盒子等
\usepackage{boxie}

%%% Local Variables: 
%%% mode: latex
%%% TeX-master: "../main.tex"
%%% End:

